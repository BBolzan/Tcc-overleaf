\subsection{Sprint 0}

\indent A Sprint 0, que começou em março, foi para mim um período de adaptação à metodologia da AGES, na qual descobri que o aprendizado é autônomo e conduzido por nós mesmos. Após a primeira reunião com os stakeholders, os AGES IV realizaram a elicitação das User Stories e nos designaram as tarefas iniciais de prototipação no Figma. Nossa equipe foi, então, dividida em dois squads: um para as telas do formulário e outro, do qual eu fazia parte, para o dashboard
administrativo.

\indent Minha primeira tarefa foi a criação da barra lateral do dashboard, um componente que planejamos reutilizar em todas as telas do painel. O início do trabalho com o Figma foi desafiador,pois eu não tinha experiência prévia com a ferramenta. Felizmente, com a ajuda de um colega AGES III, que tinha conhecimento avançado na plataforma, consegui superar essas dificuldades iniciais e me familiarizar com o processo.

\indent Após dedicar algumas horas para estudo e prática, consegui desenvolver sozinho a tela de login do dashboard, que pode ser vista na Figura 7. Optei por uma abordagem de design minimalista, mas, durante o processo, percebi que na primeira versão eu havia esquecido de incluir a funcionalidade de recuperação de senha. Identifiquei isso como um requisito importante e adicionei o botão "esqueci a senha" na iteração seguinte do protótipo.

\begin{figure}[H]
    \centering
    \small
    \caption{Mockups da tela de login do dashboard}
    \includegraphics[width=1\linewidth]{conteudo//2 - ages I//conteudo//figures/sprint-0-ages1.png}
    Fonte: \url{https://www.figma.com/design/rvV8tlcaEEW5w2hlxXKLjU/Agendamento-de-Visitas-de-Escola-ao-Museu}
    \label{fig:sprint-0-ages1}
\end{figure}

\indent Além de ter feito a tela de login, eu também ajudei minha equipe de prototipação em outras frentes, participando do design do calendário, da tela de pesquisa de agendamentos e das interfaces para a edição de informações das visitas.

\indent Paralelamente a essas tarefas de design no Figma, comecei a estudar a biblioteca React por conta própria. Meu objetivo era me preparar para uma possível alocação na equipe de frontend, já que, naquela fase do projeto, a divisão final entre as equipes de frontend e backend ainda não estava definida.