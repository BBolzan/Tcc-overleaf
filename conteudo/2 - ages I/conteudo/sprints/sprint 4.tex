\subsection{Sprint 4}

A Sprint 4 foi, para mim, a mais desafiadora, principalmente por coincidir com o final do semestre, um período com muitos trabalhos e provas de outras cadeiras. Apesar da agenda apertada, consegui organizar meu tempo para me dedicar a tarefas que eram essenciais para a entrega final do projeto. 

Uma das user stories mais importantes que pegamos era a de permitir que um administrador bloqueasse datas no calendário, registrando qual usuário realizou a ação. O que parecia uma tarefa simples, a princípio, nos apresentou um desafio: como identificar o usuário que estava bloqueando a data? A solução que eu e meu colega de squad, Pedro, elaboramos foi solicitar ao frontend que enviasse o token de autenticação do usuário (gerado pelo Keycloak) na requisição. A partir desse token, nosso backend conseguiria extrair as informações do usuário. Para viabilizar isso, criamos um \ac{dto} com os campos de data e token, o que simplificou a implementação e nos permitiu concluir a tarefa rapidamente. 

Minha segunda task na sprint foi mais simples: um endpoint getVisitLimit, que retornava o número máximo de visitantes permitidos por dia. A decisão que tomamos em equipe foi de que esse limite seria global, ou seja, o mesmo para todos os dias. Portanto, a implementação consistiu em apenas retornar o valor do atributo correspondente na tabela "calendar". 

A terceira tarefa foi mais complexa e crucial para o painel administrativo. Ela envolvia a criação de dois cards na tela inicial: um mostrando os visitantes do dia (atual vs. máximo) e outro com as estatísticas do mês (total de visitantes no mês vs. capacidade máxima mensal). Durante o desenvolvimento, enfrentei dificuldades na lógica para calcular o somatório mensal de visitantes. Para resolver o impasse, pedi ajuda ao nosso AGES IV, Lucas Silva, e trabalhamos juntos na solução em uma sessão de pair programming. Com a ajuda dele, consegui superar o desafio e finalizar a funcionalidade com sucesso antes do prazo final da entrega.