\subsection{Sprint 0}
\indent Em agosto, iniciamos o Projeto Planline, que marcou o começo da minha jornada como AGES II. Para esta nova fase, meu objetivo pessoal era me aprofundar em desenvolvimento mobile com React Native \cite{reactnative}, uma área que eu não tive a oportunidade de explorar no projeto anterior. Para acelerar meu aprendizado, comecei a desenvolver um projeto pessoal paralelo: uma calculadora de IMC para dispositivos móveis.

\indent Além dos estudos, dediquei-me ativamente à fase de design do nosso projeto. Em uma sessão de trabalho com meu colega Eduardo Garcia, colaborei na ideação do fluxo do aplicativo mobile. Minhas sugestões incluíram a adição de um polvo como imagem de fundo na tela de seleção de nomes, o que contribuiu para a identidade visual, e o posicionamento do botão "café" na horizontal, para melhorar a usabilidade da interface. Os designs do fluxo mobile nos quais participei são apresentados na Figura 12.

\begin{figure}[H]
    \centering
    \small
    \caption{Telas do fluxo mobile nas quais participei do design}
    \includegraphics[width=1\linewidth]{conteudo//3 - ages II//conteudo//figures/sprint-0-imagem-1-ages2.png}
    Fonte: \url{https://tools.ages.pucrs.br/planline/wiki/-/wikis/mockups}
    \label{fig:sprint-0-imagem-1-ages2}
\end{figure}

\indent Minha contribuição no design também se estendeu à interface web do Planline. Em uma sessão de trabalho colaborativo com meu colega Bruno Chanan, nós criamos a primeira versão da tela principal do jogo no Figma. Durante esse processo de ideação, eu propus o conceito de usar uma mesa de poker como elemento central e a figura de um "dealer" para dar mais autenticidade à experiência. A equipe gostou da ideia conceitual, embora o nosso design inicial não tenha sido aprovado, ele serviu como base para a versão atual do jogo. A primeira proposta de tela que desenvolvemos, com a ideia que evoluiu para o jogo atual, é a que apresento na Figura 13.

\begin{figure}[H]
    \centering
    \small
    \caption{Primeira proposta de design para a tela principal do jogo}
    \includegraphics[width=1\linewidth]{conteudo//3 - ages II//conteudo//figures/sprint-0-imagem-2-ages2.png}
    Fonte: \url{https://tools.ages.pucrs.br/planline/wiki/-/wikis/mockups}
    \label{fig:primeira-ideia-jogo}
\end{figure}