\section[Conclusão]{Conclusão}

\indent Ao refletir sobre minha trajetória como AGES IV no projeto Hopeful, posso afirmar com segurança que esta foi a experiência em que mais me desafiei e, paradoxalmente, em que obtive meu melhor desempenho. Cursando AGES IV simultaneamente com outras 11 disciplinas — pois deixei todas as eletivas para o último semestre — enfrentei um desafio de gestão de tempo sem precedentes na minha vida acadêmica. A pressão se intensificou especialmente na última sprint, quando precisei tirar férias do meu trabalho para conseguir me concentrar em todas as demandas do projeto e das outras disciplinas. Apesar dessa sobrecarga, consigo avaliar que desempenhei bem o papel de líder de grupo e de AGES IV, demonstrando comprometimento e responsabilidade que superaram minhas próprias expectativas.

\indent Em termos de crítica e autocrítica sobre minha atuação técnica, considero que me saí muito bem. Além de liderar a frente de backend, atuei diretamente nas tarefas mais complexas e críticas, como a implementação do JWT \cite{jwt}, a integração do sistema de autenticação, a refatoração da estrutura de pastas do Next.js \cite{nextjs}, a integração completa da geração de \ac{pdf} usando Thymeleaf \cite{thymeleaf}, e até mesmo a configuração de HTTPS usando Let's Encrypt \cite{letsencrypt} e Certbot \cite{certbot} na \ac{vps}. Muitas vezes, quando percebia que as entregas estavam em risco ou que determinadas funcionalidades não seriam concluídas a tempo, assumia a responsabilidade e implementava o que faltava, garantindo que o cliente saísse com um sorriso no rosto e a equipe com a sensação de dever cumprido. Essa proatividade técnica foi essencial para o sucesso do projeto, mas também revela um dos pontos que poderia ter sido melhor gerenciado.

\indent No que se refere às soft skills, reconheço que houve avanços e desafios. Recebi muitos feedbacks positivos nas retrospectivas sobre meu empenho, domínio do projeto e planejamento do que deveria ser feito. Fui inclusive responsável, dentro do time de AGES IV, por fazer a gestão dos próprios AGES IV — definindo quem cuidaria do quê e de quem, delegando funções conforme a professora Alessandra havia me solicitado no one-on-one. Esse reconhecimento da equipe como líder foi muito gratificante e me fez perceber o quanto estava contribuindo para o grupo. Porém, minha gestão de projeto apresentou algumas falhas significativas. No início, como apontado pela professora no one-on-one da Sprint 2, eu estava tentando fazer tudo sozinho e não envolvendo suficientemente os outros AGES IV, o que demonstrava falta de confiança nos meus colegas. Felizmente, levei esse feedback a sério e consegui melhorar nas sprints seguintes, dividindo melhor as responsabilidades.

\indent Outro ponto crítico que identifiquei foi minha gestão de artefatos e processos. Não sabia que existiam os artefatos de AGES IV até a última sprint — algo que descobri tardiamente e que gerou trabalho extra no final. Além disso, não me lembrava muito bem de como quebrar adequadamente as User Stories, e a qualidade das tasks, especialmente no início, poderia ter sido melhor. Os critérios de aceitação que escrevi na Sprint 1, por exemplo, estavam ambíguos, o que gerou confusão para o Gabriel, um AGES II que precisou de esclarecimentos adicionais. Esse tipo de problema evidencia que minha parte de soft skills voltada para gestão de projetos não estava tão desenvolvida quanto minha capacidade técnica e de liderança interpessoal.

\indent Um dos maiores problemas que enfrentei foi não ter envolvido mais a professora Alessandra nas situações em que minha cobrança sozinha não estava funcionando. Em várias ocasiões, membros da equipe demoravam excessivamente para entregar tasks simples ou subiam pull requests sem descrição e não respondiam quando chamados para conversar. Eu cobrava, mas não obtinha resultados satisfatórios. Percebi, tarde demais, que deveria ter escalado essas questões para a professora, pois apenas ela teria a autoridade necessária para resolver essas situações de forma mais efetiva. Essa foi uma lição importante sobre reconhecer os limites da minha atuação como líder de pares e saber quando buscar apoio de figuras de autoridade formal.

\indent Outro aspecto que poderia ter sido melhor gerenciado foi o controle da Wiki do projeto. Deveríamos ter cuidado melhor da documentação ao longo das sprints, registrando decisões técnicas, arquiteturais e de gestão de forma mais sistemática. A Wiki acabou ficando em segundo plano diante das demandas de desenvolvimento, o que prejudicou a rastreabilidade de informações importantes e dificultou a integração de novos membros ou a recuperação de contexto em momentos críticos.

\indent Sobre a pergunta se este foi o melhor projeto trabalhado, minha resposta é sim. Não apenas pela qualidade técnica do que entregamos — um sistema completo com autenticação JWT \cite{jwt}, geração de \ac{pdf}, integração com banco de dados PostgreSQL \cite{postgresql}, responsividade mobile, testes automatizados com 91\% de cobertura no backend, recursos de acessibilidade para pessoas com deficiência visual e deploy em ambiente de produção — mas especialmente pelo impacto social que o projeto pode gerar. Conversar com o Abner e entender a magnitude de como essa aplicação pode ajudar a salvar vidas em desastres naturais, como os que ocorreram recentemente aqui no Sul do Brasil, me fez perceber que estávamos construindo algo verdadeiramente importante. Saber que cidades poderiam ter planos de contingência melhor estruturados graças ao nosso trabalho transformou minha percepção sobre o propósito da tecnologia e do trabalho que fazemos como desenvolvedores.

\indent Esse senso de propósito intensificou minha dedicação ao projeto de uma forma que não havia experimentado em AGES anteriores. Quando você não é AGES IV, muitas vezes não se importa tanto com o resultado final da entrega, mas ao assumir essa posição de responsabilidade máxima, comecei a dar outra importância ao projeto. Cada funcionalidade incompleta, cada bug não corrigido, cada detalhe mal acabado passaram a me incomodar pessoalmente. Era eu quem mais estava se importando com o projeto e com a qualidade da entrega, e isso me motivou a fazer aquelas "horas extras" de código nas retas finais das sprints, garantindo que o cliente recebesse sempre o melhor que poderíamos oferecer. Essa experiência me fez crescer tanto profissionalmente quanto pessoalmente, desenvolvendo não apenas habilidades técnicas e de gestão, mas também valores como responsabilidade, comprometimento e empatia.

\indent Entre os aspectos positivos do projeto, destaco: a qualidade técnica do código entregue, com arquitetura bem estruturada em camadas (Controller, Service, Entity Mapper, Repository) e alta cobertura de testes; a entrega completa da aplicação em ambiente de produção do cliente, sem deixá-lo com apenas um código para contratar alguém para subir; a colaboração efetiva do time de AGES IV, especialmente após o feedback da professora sobre distribuir melhor as responsabilidades; a implementação de funcionalidades complexas como geração dinâmica de \ac{pdf}, autenticação segura com JWT \cite{jwt}, e recuperação de senha com tokens temporários; e a capacidade de lidar com imprevistos, como o adiantamento da entrega da Sprint 3, ajustando o planejamento para entregar em tempo hábil reduzido.

\indent Fazendo uma autoavaliação do meu desempenho como AGES IV, considero que me saí muito bem. Consegui equilibrar pontos fortes — como a liderança técnica, a proatividade em momentos críticos, a capacidade de resolver problemas complexos, o comprometimento com a qualidade da entrega e o impacto positivo no time — com as áreas que identifiquei para melhoria, como a gestão de artefatos e o envolvimento mais efetivo da professora em situações específicas. Reconheço essas oportunidades de crescimento e as levarei como aprendizado para experiências futuras.

\indent Concluindo, AGES IV foi muito mais do que uma disciplina de conclusão de curso — foi uma experiência transformadora que me preparou verdadeiramente para o mercado de trabalho. Os desafios enfrentados, desde gerenciar um time diverso de AGES I a IV até lidar com mudanças de escopo, imprevistos de cronograma e pressões de entrega, são reflexos diretos do que vivenciarei profissionalmente. A responsabilidade de entregar não apenas código, mas uma solução funcional que pode salvar vidas, me ensinou que tecnologia é sobre pessoas e impacto real. Saio dessa experiência com a certeza de que sou capaz de liderar projetos complexos, aprender com meus erros, adaptar-me a feedbacks e, acima de tudo, manter o foco na entrega de valor genuíno para o cliente e para a sociedade.