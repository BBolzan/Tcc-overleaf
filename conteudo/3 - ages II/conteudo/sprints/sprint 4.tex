\subsection{Sprint 4}
\indent Na Sprint 4, eu realizei diversas atividades planejadas para a entrega final. Iniciei abordando a Task 85, cujo objetivo era corrigir um problema com caracteres especiais que ocorria ao importar user stories para o menu de resumo da aplicação web. Antes da minha intervenção, o sistema não lidava corretamente com a codificação desses caracteres, o que causava bugs e a exibição de símbolos incorretos na interface. A Figura 18 ilustra o problema que ocorria.

\begin{figure}[H]
    \centering
    \caption{Tela de resumo com caracteres bugados antes da correção}
    \includegraphics[width=0.85\linewidth]{conteudo//3 - ages II//conteudo//figures/sprint-4-imagem-1-ages2.png}
    
    Fonte: \url{https://tools.ages.pucrs.br/planline/wiki/-/wikis/mockups}
    \label{fig:sprint-4-imagem-1-ages2}
\end{figure}

\indent Além da tarefa anterior, eu também executei a Task 71, que consistia na criação de um componente de toast para a exibição padronizada de mensagens de erro em todo o sistema. Atuei também em colaboração com meu colega Gabriel nas Tasks 59 e 60, cujo objetivo era desenvolver os modais de \ac{crud} (Create, Read, Update, Delete) para as seções de organizações, squads, projetos e usuários. 

\indent Em uma frente de trabalho paralela, nós realizamos a integração dos endpoints necessários para carregar as informações nesses modais. Essa atividade se conectou diretamente com a task do toast, pois utilizamos o componente que criei para exibir as mensagens de erro durante essas operações. A Figura 19 apresenta os modais desenvolvidos para o gerenciamento de organizações, projetos, squads e usuários.

\begin{figure}[H]
    \centering
    \includegraphics[width=0.45\linewidth]{conteudo//3 - ages II//conteudo//figures/sprint-4-imagem-2-ages2.png}
    \hfill
    \includegraphics[width=0.45\linewidth]{conteudo//3 - ages II//conteudo//figures/sprint-4-imagem-3-ages2.png}
    
    \vspace{0.5cm}
    
    \includegraphics[width=0.45\linewidth]{conteudo//3 - ages II//conteudo//figures/sprint-4-imagem-4-ages2.png}
    \hfill
    \includegraphics[width=0.45\linewidth]{conteudo//3 - ages II//conteudo//figures/sprint-4-imagem-5-ages2.png}
    
    \caption{Modais de \ac{crud} desenvolvidos: Organização, Projetos, Squads e Usuários}
    
    Fonte: \url{https://tools.ages.pucrs.br/planline/wiki/-/wikis/mockups}
    \label{fig:modais-crud}
\end{figure}

\indent Consegui concluir com sucesso todas as tarefas que assumi nesta sprint, entregando as funcionalidades sem erros. A parceria com o Gabriel foi especialmente produtiva; a sua grande habilidade em frontend facilitou muito nosso trabalho, e ele demonstrou uma excelente capacidade de transmitir conhecimento de forma clara.

\indent Durante o desenvolvimento, não encontrei problemas significativos. O principal ajuste técnico que precisei fazer foi no componente de toast, para que ele pudesse lidar não apenas com mensagens de erro, mas também com notificações de status — como indicar ao usuário qual seção do menu estava aberta, melhorando a navegabilidade do sistema.

\indent Uma das lições mais valiosas desta sprint foi o aprendizado sobre o componente toast, uma ferramenta com a qual eu não tinha familiaridade, e a experiência de construir modais do zero. Gostei particularmente da utilidade do toast e certamente pretendo incorporar esse componente em meus projetos futuros. Essa sprint me proporcionou um aprendizado prático extremamente útil para o meu desenvolvimento profissional.