\section[Introdução]{Introdução}
    O projeto "Agendamento de Visitas de Escola ao Museu", solicitado pelos stakeholders Simone Flores Monteiro e Marcus Vinicius Klei, consiste em uma aplicação web que redireciona os usuários do site do Museu da \ac{pucrs} para um formulário de agendamento. Desenvolvido pela equipe do projeto, o sistema armazena as informações em um banco de dados e inclui um painel administrativo (dashboard). Por meio deste painel, os funcionários do museu podem visualizar os agendamentos, bloquear datas, editar informações e autorizar as visitas.

    A necessidade desta integração surgiu para otimizar o processo atual de agendamento, que exige que as escolas preencham um formulário disponibilizado no site do museu. Atualmente, após esse preenchimento, um e-mail é enviado à equipe da secretaria, que precisa transcrever manualmente os dados para o sistema onde armazena as informações das visitas, escolas e responsáveis.

    O projeto foi executado entre março e junho de 2023, sob a orientação do professor Azriel Majdenbaum e com a participação da equipe de desenvolvimento (Figura 1)

\begin{figure}[H]
    \centering
    \small
    \caption{Time AGENDAMENTO DE VISITAS DE ESCOLA AO MUSEU}
    \includegraphics[width=1\linewidth]{conteudo//2 - ages I//conteudo//figures/foto-time-ages1.png}
    Fonte: \url{https://tools.ages.pucrs.br/agendamento-de-visitas-de-escola-ao-museu/Wiki/-/wikis/home}
    \label{fig:projeto-time}
\end{figure}