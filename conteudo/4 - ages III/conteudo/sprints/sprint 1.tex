\subsection{Sprint 1}
\indent Na Sprint 1, nosso projeto entrou em uma fase mais avançada de implementação. Os AGES IV nos dividiram em três squads, cada um responsável por uma User Story específica. Meu grupo, apelidado de "Exalta Samba", ficou encarregado do desenvolvimento frontend das telas de autenticação do sistema. Nossa responsabilidade abrangia a criação de três telas principais: a de login (com campos para email e senha), a de "esqueci minha senha" (onde o usuário insere o e-mail para receber um token de redefinição) e, por fim, a tela de "reset de senha", acessada pelo link enviado ao usuário. Enquanto isso, o squad do Marcelo Marcon ficou com o backend de cadastro e recuperação de senha, e o grupo do Henrique assumiu o desenvolvimento da homepage.

\indent Com o bom andamento da sprint, os AGES IV sugeriram que adiantássemos uma User Story futura, referente à barra lateral do painel administrativo. Além disso, como AGES III, eu e meus colegas ficamos encarregados de tarefas de infraestrutura: desenvolver o diagrama de deploy, elaborar o orçamento da AWS \cite{aws} e encaminhar essas informações ao arquiteto da AGES para a liberação da nossa máquina virtual (VM).

\indent Ao final da sprint, todos os times conseguiram concluir suas atividades. O time do Henrique reproduziu a homepage com fidelidade ao design do Figma, e o time do Marcelo entregou as funcionalidades de backend. Meu squad também teve êxito na implementação das telas de login, e acredito que conseguimos aprimorar o design original com melhores espaçamentos e com validações de campo. Isso permitiu que nós, os AGES III, integrássemos tudo na branch de desenvolvimento para realizar testes antes da apresentação.

\indent No entanto, em relação às minhas responsabilidades como AGES III, embora tenhamos concluído o diagrama de deploy, falhamos em finalizar o orçamento e em enviar a documentação para o arquiteto a tempo. Durante o processo, encontramos alguns desafios. O principal foi a indefinição sobre o serviço de e-mail que usaríamos, o que nos forçou a criar uma simulação (mock) da funcionalidade de recuperação de senha. Outro problema foi uma falha de comunicação de minha parte sobre o prazo da sprint, o que me levou a acelerar e sobrepor tarefas que já estavam com outros membros do meu time. Também reconheço que priorizamos o desenvolvimento em detrimento das tarefas de infraestrutura.

\indent Como principal lição aprendida, identifiquei a necessidade de melhorarmos nossa comunicação. Faço uma autocrítica e reconheço que não desempenhei meu papel de líder de squad da melhor forma, pois tive uma postura mais passiva. Para a próxima sprint, meu objetivo é ser mais assertivo na coordenação, acompanhando proativamente o time e oferecendo suporte. Para os próximos passos, nosso foco será finalizar as pendências de arquitetura e, para meu grupo, iniciar o desenvolvimento da User Story do questionário.