\section[Introdução]{Introdução}
\indent Para meu último projeto no memorial, nossa equipe desenvolveu a solução "Gestão de Planos de Contingência em Desastres". O projeto foi uma solicitação do cliente Abner Willian Quintino de Freitas (Hopeful), e nosso principal objetivo foi criar uma aplicação web intuitiva e segura para auxiliar na resposta e no planejamento a desastres. A necessidade do projeto surgiu da percepção de que muitos municípios e organizações carecem de ferramentas digitais integradas para gerenciar seus planos de contingência de forma colaborativa e em tempo real.

\indent A principal justificativa para o projeto é a crescente frequência e gravidade dos desastres, que demandam maior capacidade de planejamento das instituições. Como as ferramentas atuais são insuficientes, nosso desafio foi criar uma plataforma que preenchesse essa lacuna, oferecendo acessibilidade, organização e suporte à tomada de decisão durante emergências.

\indent A solução que desenvolvemos é um aplicativo web responsivo voltado para gestores públicos e coordenadores de defesa civil. No sistema, nós permitimos a construção de planos de contingência baseados na classificação de desastres (\ac{cobrade}), vinculando ações, responsáveis e recursos necessários. A plataforma conta com um painel administrativo, área de edição colaborativa e integração com mapas interativos. O escopo também incluiu funcionalidades como checklists de recursos, georeferenciamento de locais e a exportação dos planos em formatos como \ac{pdf} e DOCX.

\indent Nós desenvolvemos este projeto durante o segundo semestre de 2025, no âmbito da disciplina da AGES (turma 2JK4JK), sob a orientação da professora Alessandra Dutra e com a equipe de desenvolvimento que apresento na Figura 23.

\begin{figure}[H]
    \centering
    \caption{Fotografia da equipe do projeto (juntamente com o cliente)}
    \includegraphics[width=1\linewidth]{conteudo//5 - ages IV//conteudo//figures/foto-time-ages4.jpg}
    
    Fonte: \url{https://tools.ages.pucrs.br/gestao-de-planos-de-contingencia-em-desastres/hopeful-wiki/-/wikis/home}
    \label{fig:foto-time-ages4}
\end{figure}