\subsection{Sprint 4}
\indent A Sprint 4 foi focada em uma das entregas mais críticas do projeto: a configuração do certificado \ac{ssl}. Essa tarefa se tornou crucial após um problema na sprint anterior, onde não conseguimos demonstrar a funcionalidade do mapa (que usava a \ac{api} do Google Maps) por nossa aplicação ainda não operar em \ac{https}. Além disso, eu tinha como tarefa pendente o ajuste final do questionário, que envolvia consumir dados do backend para evitar a criação de registros duplicados de comunidades.

\indent Consegui concluir com sucesso ambas as tarefas. Para implementar o \ac{ssl}, como era um processo novo para mim, buscamos a orientação do professor Dilnei, que nos sugeriu o uso do Let's Encrypt. O primeiro passo foi registrar um domínio, o www.saudebucalquilombola.site. Em seguida, eu e o Marcelo instalamos o Nginx em nossa instância \ac{ec2}, um processo que nos apresentou desafios de memória limitada, mas que superamos com o auxílio do arquiteto da AGES, Willian. Outro obstáculo foi a necessidade de configurar um proxy reverso no Nginx para gerenciar as portas \ac{http} (80) e \ac{https} (443), redirecionando o tráfego para a porta da nossa aplicação. Após a configuração, instalamos o Let's Encrypt via Certbot e ajustamos os registros \ac{dns}. O resultado foi um sucesso: nosso site passou a funcionar com \ac{https}.

\indent Quanto à segunda tarefa, em um esforço final com os colegas Marcos e Lucas, dedicamos o feriado antes da apresentação para uma sessão intensa de bug fixing, garantindo a estabilidade do sistema. Apesar da entrega satisfatória, a sprint nos apresentou alguns desafios. Identificamos que a comunicação entre as equipes de frontend e backend ainda era um ponto a ser melhorado. Um problema inesperado também surgiu na apresentação final: percebemos uma quebra de expectativa por parte dos clientes. Acredito que eles não compreenderam que estávamos entregando um \ac{mvp} (código-fonte e funcionalidade básica), e não um sistema em produção contínua. Considero isso uma crítica construtiva à estrutura da AGES, pois lidar com essa frustração do cliente no final de um projeto tão bem-sucedido é um desafio.

\indent Pessoalmente, esta foi a sprint de maior aprendizado em infraestrutura e segurança web. Compreendi na prática como implementar certificados \ac{ssl}/\ac{tls} e o funcionamento do Nginx como servidor web e proxy reverso. Gostei imensamente de aprender esses conceitos fundamentais.

\indent Para os próximos passos, planejamos corrigir bugs menores e aprimorar a documentação na wiki do projeto para facilitar futuras manutenções.