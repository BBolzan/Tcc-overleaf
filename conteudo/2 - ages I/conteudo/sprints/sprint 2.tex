\subsection{Sprint 2}
\indent Na Sprint 2, meu squad recebeu a tarefa de estudar e integrar o Keycloak no backend do projeto. O Keycloak é uma ferramenta de código aberto voltada para o gerenciamento de autenticação e autorização, responsável por controlar credenciais de usuários e suas permissões dentro de um sistema. Essa funcionalidade é essencial para garantir a segurança e o controle de acesso adequado na aplicação que estávamos desenvolvendo.

\indent A primeira decisão técnica que enfrentamos foi definir qual abordagem utilizar para a integração. Após discussão em equipe, optamos por implementar o Keycloak via \ac{api} nativa do Nest.js, aproveitando a experiência prévia que um dos membros do squad já possuía com essa tecnologia em outra disciplina. Essa decisão se mostrou estratégica, pois nos permitiu contar com conhecimento prático desde o início da implementação.

\indent Durante a primeira semana da sprint, dediquei tempo ao estudo dos conceitos fundamentais do Keycloak. Busquei compreender não apenas sua funcionalidade básica como gerenciador de usuários, mas também aspectos mais técnicos como o processo de troca de chaves criptográficas e os mecanismos de autenticação que a ferramenta utiliza. Esse aprofundamento teórico foi importante para me preparar para contribuir de forma mais efetiva nas etapas seguintes da integração.

\indent Paralelamente aos estudos, também realizei contribuições práticas ao projeto. A equipe de frontend estava realizando testes no formulário de cadastro e identificou que o campo "NumberAddress" (número do endereço) estava configurado como obrigatório no backend. Eles solicitaram que o campo fosse alterado para opcional, considerando que existem casos reais onde endereços podem não possuir numeração. Após avaliar a solicitação, implementei o ajuste na tabela de visitas e integrei a modificação na branch de desenvolvimento, garantindo que o frontend pudesse prosseguir com seus testes.

\indent No final da semana, especificamente na sexta-feira (28/04), percebi uma oportunidade de colaboração adicional com a equipe de frontend. Até aquele momento, os desenvolvedores frontend trabalhavam apenas com seus repositórios (formulário e sistema administrativo) sem necessidade de executar o backend localmente. No entanto, com o avanço do projeto, tornou-se necessário que eles configurassem o ambiente completo para realizar testes de integração mais robustos. Dediquei parte do dia auxiliando-os na configuração do Docker e na execução do backend em suas máquinas, solucionando problemas de ambiente e garantindo que todos conseguissem subir a aplicação corretamente.

\indent Essa experiência de colaboração interdisciplinar foi valiosa não apenas por resolver um problema imediato da equipe, mas também por me permitir aprofundar ainda mais meus conhecimentos tanto no Keycloak quanto nas dinâmicas de trabalho em equipe. Ao final da sprint, me senti melhor preparado para participar mais ativamente do desenvolvimento da integração nas etapas seguintes do projeto.