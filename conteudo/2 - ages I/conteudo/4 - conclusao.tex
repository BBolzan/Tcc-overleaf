\section[Conclusão]{Conclusão}

Ao iniciar no AGES I, eu tinha uma base teórica, mas pouco conhecimento prático sobre conceitos essenciais como a construção de uma \ac{api}, o protocolo \ac{http}, a integração entre frontend e backend, e o desenvolvimento de um \ac{crud}. As disciplinas anteriores do curso não preparam para a intensidade de um ambiente ágil como o da AGES, onde a expectativa de desenvolvimento começa já na primeira sprint. Após completar esta primeira fase, considero a AGES a disciplina mais importante do curso de Engenharia de Software.

As duas primeiras sprints foram, na minha visão, as mais difíceis. Como eu não tinha a mesma experiência prévia de outros colegas, precisei me dedicar intensamente para nivelar meu conhecimento e acompanhar o ritmo do projeto. O ponto de virada ocorreu na Sprint 3, quando mudei minha abordagem para uma postura mais proativa e sem medo de errar. Foi nesse momento que percebi que a barreira era mais uma insegurança pessoal do que uma dificuldade técnica real, e que sair da zona de conforto era fundamental para o meu desenvolvimento.

Um dos maiores desafios da AGES é a conciliação com as outras disciplinas. Por ser uma cadeira com peso de TCC, a dedicação exigida em tempo e esforço é muito grande. Como aprendizado, para os próximos semestres, planejarei minha grade curricular de forma a reduzir o número de outras disciplinas cursadas simultaneamente.

O ambiente colaborativo e a união da equipe foram fatores essenciais para minha motivação e sucesso no projeto. O bom relacionamento que construímos, inclusive com encontros após o horário de aula para discutir o projeto, foi fundamental para manter o engajamento de todos em busca dos nossos objetivos.

Fazendo uma autoavaliação, acredito que meu desempenho no semestre corresponde a uma nota 9. Minha performance nas duas primeiras sprints foi mais baixa (em torno de 8), pois minha contribuição em código foi limitada. No entanto, nas sprints 3 e 4, acredito ter alcançado um desempenho excelente (nota 10), pois não só desenvolvi tarefas essenciais como também auxiliei meus colegas com a configuração do ambiente de desenvolvimento. 

A lista de aprendizados técnicos nesta primeira fase da AGES é extensa e inclui desde o funcionamento de uma \ac{api} e os status codes do protocolo \ac{http}, até o uso de tecnologias como Prisma, Node.js, o framework Nest.js, Swagger, Docker e \ac{aws}. Foi, sem dúvida, uma experiência de aprendizado muito intensa e valiosa.