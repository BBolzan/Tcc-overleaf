\chapter[APRESENTAÇÃO DA TRAJETÓRIA DO ALUNO]{APRESENTAÇÃO DA TRAJETÓRIA DO ALUNO}

\indent Minha jornada na tecnologia sempre foi movida pela curiosidade e pelo desejo de construir soluções. Essa paixão me levou à Engenharia de Software e, mais recentemente, à \ac{ages} IV, onde tenho vivenciado o que considero a essência do desenvolvimento: trabalho colaborativo, aprendizado constante e impacto real. Foi durante a graduação, como monitor de Lógica para Computação, que descobri que tecnologia vai além de código – é sobre pessoas, sobre ensinar, compartilhar conhecimento e crescer junto. Essa experiência moldou minha visão profissional e me ensinou que os melhores projetos nascem da combinação entre habilidades técnicas e conexões humanas genuínas.

\indent Essa visão se consolidou quando dei meus primeiros passos profissionais como Desenvolvedor de Sistemas Júnior, trabalhando com a stack clássica da web: PHP \cite{php}, JavaScript \cite{javascript}, HTML \cite{html} e \ac{css}. Ali construí uma base sólida em desenvolvimento web e aprendi os fundamentos do trabalho em equipe corporativo. A busca por expandir horizontes me levou à Engenharia de Dados, onde compreendi a importância estratégica dos dados nas organizações modernas.

\indent Atualmente, retornei para a área de desenvolvimento, atuando com uma stack moderna composta por Python \cite{python} e React \cite{react}, em uma posição de nível quase pleno. Esta trajetória diversificada me proporcionou uma visão ampla da tecnologia, desde o desenvolvimento front-end e back-end até a engenharia de dados, permitindo-me atuar de forma versátil e contribuir significativamente nos projetos em que me envolvo.
    