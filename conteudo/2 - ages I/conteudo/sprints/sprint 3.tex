\subsection{Sprint 3}
A terceira sprint do projeto foi um ponto de virada para mim, pois foi quando eu realmente comecei a contribuir com código e a ajudar meu squad a resolver os problemas das nossas tasks. A motivação para essa mudança veio após uma conversa individual (one-to-one) com o professor orientador, Azriel, na qual refleti que eu ainda não havia conseguido agregar valor técnico ao projeto como gostaria. Decidido a mudar esse cenário, dediquei horas a estudar e analisar o código desenvolvido pelos meus colegas de backend para me aprofundar na base do projeto.

Após esse período de estudo, senti-me mais confiante e assumi minha primeira task no Trello, que consistia em criar o endpoint getVisitasPendentes. A funcionalidade exigia uma requisição sem parâmetros ao visit/controller, que deveria retornar uma lista com todas as visitas com status "pendente". Consegui completar essa tarefa de forma ágil e, em seguida, solicitei ao meu AGES IV um code review do meu código para poder integrá-lo à branch de desenvolvimento (dev). Como estávamos perto do final da sprint, eu queria continuar contribuindo para resolver as tarefas restantes.

Com isso em mente, assumi a implementação de dois outros métodos similares: getVisitasConfirmadas e getVisitasReprovadas. A lógica era muito parecida com a da tarefa anterior, sendo necessário apenas alterar o enum de status em cada método. Após finalizar e submeter essas duas tarefas para revisão, percebi que ainda tinha tempo disponível, pois já havia concluído minhas outras atividades acadêmicas. Comuniquei-me com o squad e assumi uma última task da sprint: o desenvolvimento do método getVisitaById. Essa última tarefa se mostrou um pouco mais complexa, pois exigia que o backend recebesse um ID como parâmetro na requisição. O principal desafio técnico foi a necessidade de criar um \ac{dto} (Data Transfer Object) para capturar corretamente o ID enviado no corpo (body) da requisição feita pelo frontend.