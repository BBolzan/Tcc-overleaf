\section[Conclusão]{Conclusão}
\indent Ao final do meu período na AGES II, saio com um conhecimento técnico consolidado, sabendo como criar e utilizar o banco de dados Firebase \cite{firebase}, desenvolver um aplicativo móvel com React Native \cite{reactnative} e Expo \cite{expo}, e construir um frontend web com React \cite{react}. Tão importante quanto o aprendizado técnico foram as amizades que construí ao longo deste projeto.

\indent Acredito que a AGES II foi particularmente desafiadora devido à alta exigência dos três stakeholders, que buscavam um nível de excelência no projeto. A frequência com que a nossa stakeholder e profissional de \ac{ux}, Letícia, revisava nosso trabalho e sugeria alterações nos protótipos foi uma evidência disso, o que nos incentivou a buscar um alto padrão de qualidade.

\indent Outro desafio significativo para mim foi ter começado em um novo emprego em setembro, no final da Sprint 2, o que exigiu de mim um grande esforço para conciliar as novas responsabilidades com as demandas do projeto. Antes de começar a trabalhar, eu e meus colegas tínhamos o hábito de realizar sessões noturnas de pair programming pelo Discord, uma prática de muito aprendizado. Com o início do trabalho presencial, precisei adaptar minha rotina, mas fiz um esforço para participar dessas sessões nas semanas finais das sprints.

\indent Sou extremamente grato à AGES pela experiência e, em especial, aos nossos AGES IV, Bruno Chanan e Lucas Suzin, que sempre foram muito prestativos. Também quero expressar minha gratidão ao nosso colega AGES I, Gabriel Bohn. Mesmo estando em uma fase anterior, ele demonstrou um amplo conhecimento em React e foi fundamental ao me ensinar desde os conceitos básicos até o nível intermediário do framework. Essa troca de conhecimento com o Gabriel me fez refletir sobre o verdadeiro significado da AGES: o aprendizado é recíproco e não depende da hierarquia de fases. Aprendi que, independentemente de ser AGES I, II, III ou IV, sempre haverá algo novo a aprender e a ensinar.

\indent Ao contrário da minha experiência no AGES I, onde fiquei restrito ao backend, no projeto Planline tive a liberdade de atuar em diferentes frentes, como backend, frontend e banco de dados, o que enriqueceu muito minha experiência. Concluo minha jornada na AGES II com um grande sentimento de orgulho pelo meu progresso e pelas amizades que construí. Mesmo que não tenhamos conquistado o prêmio de melhor projeto, tenho a certeza de que entregamos um trabalho de qualidade que deixou o cliente satisfeito.