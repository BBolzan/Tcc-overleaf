\subsection{Sprint 3}
\indent Na Sprint 3, nossa equipe enfrentou novos desafios de desenvolvimento. Eu fiquei responsável pela Task 52, cujo objetivo era implementar a funcionalidade de seleção de nome de usuário no aplicativo mobile. A tela pela qual fiquei responsável nesta tarefa é a que apresento na Figura 17.

\begin{figure}[H]
    \centering
    \caption{Tela de seleção de nome de usuário no aplicativo mobile}
    \includegraphics[width=0.35\linewidth]{conteudo//3 - ages II//conteudo//figures/sprint-3-imagem-1-ages2.png}
    
    Fonte: Captura de tela do projeto em execução
    \label{fig:sprint-3-imagem-1-ages2}
\end{figure}

\indent A implementação da Task 52 me apresentou alguns desafios técnicos. O primeiro foi aprofundar minha compreensão sobre o conceito de Contexto em React, que era novo para mim. Além disso, a integração com o Firebase se mostrou complexa, especialmente na lógica para alterar o ID do usuário no jogo após a seleção do nome.

\indent Somado a isso, surgiu uma demanda de ajuste não planejada: a necessidade de alterar duas tabelas do nosso banco de dados (organizações e squad). Essa mudança foi crucial, pois precisamos implementar a lógica de negócio que define que uma organização pode ter múltiplos squads, mas um squad deve pertencer obrigatoriamente a uma única organização.

\indent Pessoalmente, esta foi uma sprint complicada, na qual enfrentei desafios externos, incluindo dificuldades pessoais e novas responsabilidades no trabalho, que impactaram minha dedicação à faculdade. No entanto, apesar desses contratempos, o período foi de grande aprendizado. As dificuldades me forçaram a aprofundar meu entendimento sobre React e Firebase. No fim, mesmo com os problemas, considero que essas experiências foram valiosas e contribuíram significativamente para a expansão dos meus conhecimentos técnicos.