\subsection{Sprint 0}
\indent A Sprint 0 marcou o início da minha jornada como AGES III no projeto da "Cartilha Interativa de Saúde Bucal para Comunidades Quilombolas". Minha função evoluiu para um papel mais estratégico, e eu fiquei responsável por definir a arquitetura e as bases técnicas que guiariam todo o desenvolvimento futuro.

\indent Logo no início, assumi a responsabilidade de conduzir as discussões com os membros AGES II para a escolha do banco de dados. Após analisarmos juntos as necessidades do projeto, chegamos ao consenso de que o PostgreSQL \cite{postgresql} seria a melhor opção por ser um banco relacional robusto e completo. Além disso, fiquei encarregado de definir a tecnologia do backend e, após avaliar algumas alternativas, optei pelo NestJS \cite{nestjs} com Prisma \cite{prisma} como ORM. Para o frontend, confirmamos a utilização de React \cite{react} com Vite \cite{vite}, conforme já havíamos alinhado.

\indent Uma das minhas tarefas mais importantes foi configurar a infraestrutura inicial do projeto. Desenvolvi os boilerplates de frontend e backend para criar uma base sólida para a equipe. No frontend, implementei o Tailwind CSS \cite{tailwind} para facilitar a criação de interfaces responsivas e, no backend, adicionei o Swagger \cite{swagger} para a documentação automática de rotas. Para garantir um ambiente de desenvolvimento consistente para todos, configurei a infraestrutura em containers Docker \cite{docker} e criei scripts de shell para simplificar o processo de inicialização, pensando especialmente nos colegas com menos experiência. Também gerenciei os acessos ao GitLab \cite{gitlab}, orientando os membros que tiveram problemas de acesso a buscar suporte da equipe da AGES.

\indent Durante esse processo, enfrentei alguns obstáculos. A sobrecarga de tarefas técnicas me impediu de participar ativamente do design inicial no Figma \cite{figma} como eu gostaria. Tive também desafios técnicos, como problemas de compatibilidade com a versão mais recente do Tailwind CSS \cite{tailwind} e conflitos de porta no ambiente Docker \cite{docker} em algumas máquinas da equipe, que precisei ajudar a resolver. Uma lição importante que aprendi foi sobre a necessidade de comunicar melhor minhas limitações de tempo e a carga de trabalho aos AGES IV.

\indent Mesmo com os desafios, a experiência foi extremamente enriquecedora para o meu conhecimento técnico. Como próximo passo, fui designado líder do squad responsável por desenvolver a tela de login e sua lógica de backend. Já iniciei o desenvolvimento da interface, embora esteja enfrentando alguns desafios com o design. Noto que a equipe ainda precisa de um pouco mais de engajamento, e acredito que um incentivo dos AGES IV poderia ajudar o projeto a avançar no ritmo necessário.