\subsection{Sprint 2}
\indent Na Sprint 2, nossas atividades previstas incluíam a retrospectiva da sprint anterior, o one-on-one com a professora Alessandra (essencial para receber feedback sobre o relatório e sobre nossa atuação como líderes), e a entrega do coração da aplicação para o cliente. O entregável principal seria a criação dos cenários, a listagem de planos para o administrador, a edição de cenários pelo administrador e o preenchimento de cenários pelo usuário. Como AGES IV responsável pelo backend, fiquei encarregado de quebrar todas as tarefas desta frente, enquanto o Thiago e o Henrique cuidavam do frontend.

\indent Primeiramente, precisávamos realizar a retrospectiva. Na sprint anterior, a professora Alessandra havia solicitado que trouxéssemos uma proposta "divertida", fugindo dos padrões das retrospectivas tradicionais. Após pesquisar, encontrei um formato bem criativo com o tema de Pokémon. Nessa dinâmica, o que tinha sido bom na sprint era classificado como "Pokémon Lendário", o que tinha sido ruim ficava como "Equipe Rocket", as pessoas destaque eram os "Mestres Pokémon", e aqueles que precisavam de ajuda eram o "Magikarp" — o pokémon que tem a primeira evolução muito fraca, mas que, quando evolui, se torna muito forte.

\indent Após a retrospectiva, tive meu one-on-one com a Alessandra, no qual recebi diversos elogios sobre estar desempenhando muito bem o papel de líder, puxando o time, ajudando os outros e demonstrando interesse no projeto. Porém, recebi um feedback construtivo importante: estava tentando fazer tudo sozinho e muitas vezes não envolvendo os outros AGES IV. De fato, percebi que até ali estava me faltando um pouco de confiança nos meus colegas. Levei esse feedback muito bem para as próximas sprints, nas quais conseguimos nos dividir melhor e deixar todos participarem, sem que eu ficasse querendo abraçar tudo.

\indent Sobre o projeto, fui o responsável por quebrar as tasks do backend. Primeiramente, criei a task de inserção da migration de \ac{cobrade}, que consistia apenas em pegar o \ac{pdf} de cobrades, criar uma tabela no banco e adicionar as migrations. Como essa era uma tarefa muito fácil, delegquei para as duplas do backend que considerava mais inexperientes. Junto com essa, criei também a task de GET de todas as COBRADES, um GetAll simples, e como estava relacionada à tarefa anterior, atribuí à mesma dupla. Em seguida, criei a task mais importante da sprint: a criação do cenário. Essa era a mais difícil, pois envolvia grande complexidade de lógica e conhecimento profundo do Spring e dos plugins de banco. Atribuí para minha melhor dupla em termos de conhecimento técnico, pois sabia que demoraria e confiava neles para resolver. A edição de cenário ficaria em stand-by até a conclusão da criação, então criei essa task como bloqueada. Inicialmente, tínhamos a ideia de implementar a deleção de cenário, mas acabei removendo da sprint, pois não fazia sentido para o projeto excluir cenários — validei isso com o Abner na apresentação, e ele concordou. Mantivemos apenas a deleção de tarefas. Essa sprint foi bem difícil, e eu, André, Thiago e Henrique tivemos que fazer bastante horas de código no final para salvar a entrega.

\indent O maior problema que enfrentamos foi a demora de alguns membros para entregar tasks muito simples. Quando os chamava para perguntar sobre o progresso, recebia apenas a resposta de que estavam trabalhando nisso. Outro problema crítico ocorreu durante o code freeze: a equipe subiu pull requests sem descrição nenhuma sobre o andamento das suas tasks e, quando chamávamos para conversar, não recebíamos respostas. Isso me forçou, no final da sprint, a desenvolver uma tela inteira sozinho — a tela de preenchimento de cenário pela visão do usuário.

\indent Como lições aprendidas, após o one-on-one, compreendi que devo deixar mais espaço para meus colegas AGES IV trabalharem, que não preciso abraçar tudo sozinho, pois não vou dar conta. O grupo de AGES IV precisa estar unido. Também percebi, em retrospectiva, que deveria ter conversado com a professora sobre essas demoras nas entregas de tasks, pois apenas nossa cobrança não estava funcionando.

\indent Para os próximos passos, vamos focar no verdadeiro coração da aplicação: a geração do \ac{pdf} e a feature de pesquisa para usuário não logado.