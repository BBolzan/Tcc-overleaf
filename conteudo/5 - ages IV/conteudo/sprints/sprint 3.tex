\subsection{Sprint 3}
\indent Para essa sprint, tínhamos prometido o verdadeiro coração do projeto: a geração do \ac{pdf} para salvar o plano de contingência, que era o objetivo final da aplicação. Junto com essa funcionalidade, também foi prometida a US de Pesquisa de Planos Web na parte do usuário, com a tela responsiva para mobile. Além disso, ficamos comprometidos a fazer o deploy no ambiente do Abner, na Kinghost. Por fim, precisávamos realizar a sprint review sobre a Sprint 2, também com o formato criativo de histórias. Após a apresentação anterior, o Abner pediu que adicionássemos um footer em todas as páginas igual ao do site dele e solicitou uma flag para publicar e despublicar cenários — a lógica seria que apenas cenários publicados poderiam ter o \ac{pdf} baixado, pois os planos são um conjunto de cenários (cobrades).

\indent Começamos a sprint com a retrospectiva, escolhendo o tema de viagem ao espaço. O que "foi bom na sprint" foram as "Estrelas", o que "foi ruim" ficou como "Meteoros", os alunos destaque foram os "Astronautas Destaques", e quem precisava de ajuda ficou como "Ajustes no Foguete". Durante essa retrospectiva, tivemos insights importantes: o pessoal havia entendido de forma ligeiramente incorreta o conceito de cenário. O Abner explicou que tudo era modulado ao redor das cobrades, mas descobrimos que existem casos como terremotos, que são uma junção da cobrade de tremor de terra e tsunami. No jeito que estávamos pensando, não teríamos como juntar duas cobrades diferentes em um PDF. Para ajustar isso, definimos que o título do PDF seria sempre baseado na coluna de subgrupo da tabela cobrade. Também precisamos reverter algumas migrations que havíamos removido, como a coluna grupo da cobrade, que diferencia entre eventos tecnológicos e naturais.

\indent Fiquei responsável por quebrar as tasks do backend. Criei a tarefa de criação do template em HTML \cite{html} do \ac{pdf}, pois encontramos uma biblioteca do Spring Boot \cite{spring} chamada "Thymeleaf" \cite{thymeleaf}, que gera PDFs baseados em HTML \cite{html}. Atribuí essa task ao Gabriel e ao Eduardo Barcellos, dois AGES II. Outra task importante foi a "GDS-058: Criar rota no backend para retornar cenários prontos de uma cidade ou de uma cobrade específica". Essa rota deveria ser desprotegida (sem necessidade de login), pois o download do \ac{pdf} é público e acessível a todos, já que os dados são de consumo geral.

\indent Uma semana antes da apresentação prevista, o Abner avisou que não poderia comparecer, então a Alessandra decidiu adiantar a apresentação. No final de semana anterior, não tínhamos muita coisa pronta: os responsáveis pelo \ac{pdf} haviam feito uma rota para gerar o documento, e o Ravel, responsável pela rota que pega o cenário baseado na cidade ou cobrade, havia feito outra, mas ninguém havia integrado as duas. Tive que fazer um grande esforço para integrar tudo e salvar a apresentação, fazendo bastante horas de código. Consegui fazer tudo funcionar, embora o \ac{pdf} ainda quebrasse algumas linhas. Como havíamos avisado ao Abner que era uma entrega parcial com alguns bugs, felizmente esse foi o único problema na apresentação.

\indent Sobre os problemas durante a sprint, o maior foi com a task de geração do \ac{pdf}. Coloquei os dois AGES II, Gabriel e Eduardo Barcellos, porém eles não se comunicaram: um assumiu que o outro faria e vice-versa. Como eram dois AGES II, achei que podia contar com eles para essa tarefa complexa, mas tive que falar com ambos separadamente para entender o problema. No final, eu mesmo tive que tocar essa task na reta final, senão não entregaríamos. Outro problema foi o adiantamento da entrega. Inicialmente, decidiríamos gravar um vídeo para o Abner, mas a Alessandra sugeriu adiantarmos a apresentação para a quarta-feira anterior, já que era uma entrega-chave do projeto. Acabei preferindo assim, mesmo com o esforço muito grande de nós AGES IV e dos AGES III para ajustar tudo a tempo.

\indent Também tivemos problemas com o deploy. Desde o início, a professora sugeriu não usarmos a AWS \cite{aws} da AGES, mas sim um ambiente que já ficaria para o Abner. Nenhum de nós, AGES IV ou AGES III, já havia feito deploy em ambiente de cloud fora da AWS \cite{aws}, então estávamos perdidos: tínhamos que procurar o melhor preço, entender a plataforma e lidar com toda a configuração. Antes dessa sprint, fazíamos deploy em plataformas open source apenas para visualização e testes do Abner, mas como nessa sprint faríamos a virada de chave para o \ac{vps} na Kinghost, com o adiantamento da entrega, não conseguimos configurar tudo a tempo. Demos prioridade às User Stories e deixamos a infraestrutura para a próxima sprint.

\indent A maior lição dessa sprint foi que acontecem muitos imprevistos. Nossa entrega tinha um dia planejado e foi adiantada, então tivemos que ajustar o planejamento para manter o escopo em um tempo hábil menor — foi uma boa simulação do mercado de trabalho, onde esse tipo de imprevisto acontece frequentemente. Outra lição foi que deveria ter envolvido mais a Alessandra nas cobranças aos alunos, pois minha cobrança sozinha não estava sendo suficiente.

\indent Para os próximos passos, teremos a última sprint, com pequenas correções e algumas tasks de baixa complexidade. Vamos tentar alcançar 100\% de cobertura de testes no backend e tornar todas as telas responsivas.