\subsection{Sprint 0}
\indent A Sprint 0 marcou o início da minha jornada como AGES IV neste projeto. Minhas responsabilidades foram estratégicas desde o começo, focando em me reunir com o cliente para definir o escopo inicial, organizar os squads para o trabalho de prototipação no Figma \cite{figma}, criar as User Stories (US) e preparar e conduzir a primeira apresentação para o cliente. Além disso, tínhamos a importante tarefa de dar suporte e mentoria aos AGES III.

\indent O maior desafio que enfrentamos foi, sem dúvida, a complexidade da regra de negócio. O escopo inicial era muito aberto e confuso, o que exigiu de nós, AGES IV, várias sessões de discussão para refinar e definir o que seria de fato desenvolvido. Nesse ponto, o auxílio da professora Alessandra foi fundamental, pois ela nos guiou nas primeiras cerimônias e nos ajudou a negociar uma redução do escopo inicial que o cliente esperava.

\indent Durante a execução, considero que tivemos grande sucesso. Conseguimos entregar todos os artefatos planejados, e o feedback do cliente, Abner, sobre os fluxos de tela e o planejamento foi muito positivo. Um aprendizado importante ocorreu quando ele propôs refazer o design de uma das telas; embora a equipe tivesse outras preferências, seguimos sua visão, reforçando a máxima da AGES de que o cliente tem a palavra final. Outro desafio foi a ausência de um AGES III, que estava de férias, o que exigiu uma performance excepcional do André para cobrir a demanda. Isso me serviu como uma autocrítica: a necessidade de um "herói" geralmente indica uma falha de planejamento da liderança.

\indent Ao final da sprint, na nossa primeira retrospectiva, fiquei muito satisfeito ao receber feedbacks positivos da equipe, que me destacaram pela organização e por estar sempre presente para tirar dúvidas. Fico feliz em ver que meu engajamento está sendo percebido. Gostaria também de elogiar o AGES II Ravel e o AGES III André, que se destacaram pelo empenho em questionar e entender a complexa regra de negócio, nos ajudando a pensar e a evoluir. Mesmo o trabalho no Figma \cite{figma} não sendo o preferido de muitos, toda a equipe se empenhou, e no final, eu, o André e o Mazzocatto finalizamos a montagem dos fluxos para a apresentação.

\indent Como lição aprendida e próximos passos, nós quebramos as User Stories em tasks técnicas e as atribuímos aos membros da equipe. Além disso, implementamos um sistema de "apadrinhamento" para mentoria, uma estratégia que vamos avaliar e ajustar conforme necessário. Como uma observação final e um ponto de atenção para o futuro, notei nesta sprint inicial uma forte preferência de alguns membros da equipe por tarefas de backend, com uma certa resistência ao desenvolvimento frontend. Acredito que a filosofia da AGES incentiva a multidisciplinaridade, e é importante que todos se desenvolvam em diferentes áreas. Este é um ponto que, como AGES IV, pretendo acompanhar de perto nas próximas sprints para incentivar o crescimento completo da equipe.