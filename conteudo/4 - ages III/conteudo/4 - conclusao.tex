\section[Conclusão]{Conclusão}
\indent Concluo meu período na AGES III com um sentimento de grande evolução, tanto no conhecimento técnico quanto na minha capacidade de liderança. O projeto da "Cartilha Interativa de Saúde Bucal para Comunidades Quilombolas" foi, sem dúvida, fundamental para essa maturação. Acredito que a AGES III foi a experiência mais desafiadora da minha formação até agora, principalmente pela responsabilidade de definir a arquitetura técnica de um projeto do zero. Ao contrário das fases anteriores, onde eu executava tarefas, agora eu era responsável por criar as bases sobre as quais toda a equipe trabalharia.

\indent Um dos meus maiores desafios foi aprender a equilibrar as responsabilidades técnicas com a liderança de equipe. Na Sprint 0, sobrecarreguei-me ao tentar centralizar a configuração dos boilerplates, a definição das tecnologias e a gestão de acessos. A lição mais importante que aprendi com isso foi que delegar e comunicar minhas limitações é tão crucial quanto ter o conhecimento técnico. Outro momento difícil foi o rollback que sofremos na Sprint 3. Ver uma semana de trabalho da equipe ser perdida foi frustrante, mas a experiência reforçou a importância de termos backups robustos e políticas de versionamento rigorosas.

\indent Por outro lado, os momentos mais gratificantes vieram das conquistas técnicas complexas. Senti uma grande satisfação ao configurar o certificado SSL, implementar o pipeline de deploy automático no GitLab \cite{gitlab} e resolver os problemas de integração do questionário, especialmente por serem tecnologias que eu nunca havia utilizado antes. Além disso, trabalhar em um projeto com um propósito social tão claro, que poderia impactar positivamente a vida de pessoas, me deu uma motivação muito maior do que em projetos puramente acadêmicos.

\indent Tecnicamente, este projeto me tornou um desenvolvedor muito mais completo. Aprendi sobre configuração de infraestrutura em nuvem, dominei o NestJS \cite{nestjs} com Prisma \cite{prisma}, configurei pipelines de CI/CD e resolvi problemas complexos de CORS e \ac{https}. Também evoluí minha comunicação técnica, aprendendo a adaptar minhas explicações para os diferentes níveis de experiência da equipe.

\indent Sou extremamente grato aos stakeholders, Márcio e Patrícia Grossi, pela paciência durante nosso aprendizado, e ao professor Dilnei Venturini, pelo suporte técnico fundamental, especialmente a dica sobre o Let's Encrypt. Agradeço também aos meus colegas de equipe, Henrique e Marcelo, parceiros essenciais na infraestrutura, e a todos os membros que demonstraram uma evolução impressionante. Uma lição final veio da gestão de expectativas com o cliente, que me ensinou sobre a importância de alinhar o escopo de um MVP desde o início, um aprendizado que levarei para toda a minha carreira.

\indent Após essa jornada, percebo que o propósito da AGES III é exatamente este: não apenas liderar tecnicamente, mas também formar outros desenvolvedores, tomar decisões arquiteturais complexas e gerenciar projetos reais com impacto social. Saio da AGES III orgulhoso do meu crescimento, sabendo que, apesar de todas as dificuldades, entregamos um produto funcional que atende às necessidades dos nossos stakeholders. Este projeto confirmou meu desejo de trabalhar com tecnologia que tenha um propósito e me deu a confiança para assumir novas posições de liderança técnica no futuro. A AGES III foi, definitivamente, um divisor de águas na minha formação.