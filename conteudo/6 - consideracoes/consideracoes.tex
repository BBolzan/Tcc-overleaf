\chapter[Considerações Finais]{CONSIDERAÇÕES FINAIS}

\indent Este memorial reuniu as experiências e aprendizados vivenciados ao longo dos quatro projetos desenvolvidos na Agência Experimental de Engenharia de Software (AGES I a IV). Cada fase contribuiu de maneira singular para a construção de competências técnicas e interpessoais, consolidando minha formação e oferecendo uma visão abrangente sobre os processos e desafios inerentes à prática da Engenharia de Software.

\indent Ao refletir sobre minha trajetória completa, percebo uma evolução profissional e pessoal que superou minhas expectativas iniciais. Quando iniciei minha jornada na AGES, tinha apenas conhecimento teórico e pouca experiência prática. Hoje, concluo essa etapa como um profissional completo, capaz de liderar equipes, tomar decisões arquiteturais complexas e entregar soluções que geram impacto real na sociedade. A AGES foi, sem dúvida, a disciplina mais transformadora do curso, ensinando-me que Engenharia de Software vai muito além de escrever código — trata-se de resolver problemas reais, trabalhar colaborativamente, gerenciar expectativas de clientes, adaptar-se a imprevistos e, acima de tudo, entregar valor genuíno.

\indent Ao longo dos projetos, desenvolvi tanto competências técnicas quanto interpessoais. Em termos técnicos, consolidei o domínio de desenvolvimento backend e frontend, práticas de arquitetura de software, metodologias ágeis, infraestrutura em nuvem e ferramentas de DevOps. Mais importante ainda, exercitei soft skills fundamentais como gestão de tempo, comunicação em equipe, liderança e delegação de responsabilidades. A conciliação das atividades de AGES IV com 11 disciplinas simultâneas foi um dos maiores desafios de gestão de tempo que enfrentei, mas me preparou para lidar com múltiplas prioridades no ambiente profissional. Aprendi a comunicar ideias técnicas para diferentes níveis de experiência, adaptar-me a feedbacks e reconhecer os limites da minha atuação como líder de pares.

\indent Em cada projeto houve contribuições das quais me orgulho. No projeto Hopeful desenvolvido na AGES IV, a entrega de um sistema completo em ambiente de produção do cliente, com 91\% de cobertura de testes no backend e funcionalidades complexas como geração de PDF e autenticação segura, foi motivo de grande satisfação. A sinergia da equipe e o comprometimento com a qualidade final demonstraram a maturidade alcançada ao longo da trajetória. Na AGES III, a configuração do certificado SSL e a implementação do pipeline de deploy automático foram conquistas técnicas que exigiram pesquisa e aprendizado contínuo. Na AGES II, a versatilidade de atuar em diferentes frentes — backend, frontend e banco de dados — enriqueceu profundamente minha experiência.

\indent Reconheço que houve desafios ao longo da jornada que representam oportunidades de melhoria. A gestão de artefatos de AGES IV, que descobri tardiamente, gerou trabalho extra no final do projeto. A qualidade inicial das tasks e critérios de aceitação melhorou ao longo do tempo, mas começou com ambiguidades que poderiam ter sido evitadas. A manutenção da Wiki dos projetos frequentemente ficava em segundo plano diante das demandas de desenvolvimento. Em termos técnicos, débitos como nomes de colunas em português que deveriam ter sido em inglês desde o início evidenciaram a importância de boas práticas desde a primeira sprint. Uma alocação de tempo mais rigorosa e o envolvimento mais efetivo dos professores em situações onde minha cobrança sozinha não funcionava poderiam ter reduzido o retrabalho e melhorado ainda mais a fluidez das interações.

\indent Comparando o início da minha trajetória na AGES com o momento atual, percebo uma transformação significativa. De alguém que entendia Engenharia de Software como simplesmente escrever código e implementar funcionalidades, hoje compreendo com clareza que a disciplina é muito mais ampla e complexa. Engenharia de Software é sobre pessoas — entender necessidades de clientes, comunicar-se efetivamente com equipes, gerenciar expectativas, formar outros desenvolvedores e construir soluções que geram impacto real. Essa jornada trouxe não apenas conhecimento técnico, mas também coragem para assumir responsabilidades e sabedoria para lidar com a incerteza de projetos de software. Aprendi que tecnologia sem propósito é apenas código. Os projetos que mais me marcaram foram aqueles com impacto social claro: a Cartilha Interativa de Saúde Bucal para Comunidades Quilombolas, que poderia melhorar a qualidade de vida de pessoas historicamente marginalizadas; e o Hopeful, que poderia salvar vidas em desastres naturais como os que ocorreram recentemente no Sul do Brasil. Esses projetos transformaram minha percepção sobre o propósito da tecnologia e confirmaram meu desejo de trabalhar com soluções que façam diferença real.

\indent A AGES representa uma oportunidade ímpar para estudantes aplicarem conceitos em contextos reais. Como críticas construtivas e sugestões de melhoria para a disciplina, destaco três pontos principais. Primeiro, a \textbf{padronização dos entregáveis}: uma das maiores dificuldades que enfrentei como AGES IV foi descobrir apenas na última sprint que havia artefatos específicos de gerenciamento de projeto (Plano de Comunicação e Plano de Riscos) que deveriam ser elaborados. Em nenhuma das minhas AGES anteriores isso havia sido solicitado, o que demonstra falta de padronização entre os projetos. Sugiro que a disciplina estabeleça claramente quais são os entregáveis obrigatórios para cada fase (AGES I, II, III e IV) e que essa informação seja comunicada logo no início do semestre, evitando surpresas de última hora e permitindo que os alunos se planejem melhor ao longo das sprints.

\indent Segundo, a \textbf{participação dos AGES IV na avaliação dos alunos}: como líder de grupo, acompanhei de perto o desempenho de cada membro da equipe ao longo das sprints. Observei quem se dedicava, quem tinha dificuldades, quem não comparecia às dailies, quem não respondia quando chamado, e quem entregava tasks com qualidade. Acredito que os AGES IV poderiam estar envolvidos no processo de avaliação dos alunos, não para "bater o martelo" na nota final, mas para participar de uma votação ou fornecer um parecer detalhado sobre o desempenho individual de cada membro. Essa dinâmica daria maior poder e responsabilidade aos AGES IV, reconhecendo formalmente o papel de liderança que já exercemos na prática. Além disso, tornaria a avaliação mais justa e precisa, pois o AGES IV está presente no dia a dia do projeto e tem uma visão privilegiada do comprometimento e evolução de cada aluno.

\indent Terceiro, a \textbf{gestão de expectativas e comunicação}: sugiro que haja encontros mais frequentes entre AGES IV e professores ao longo das sprints, não apenas os one-on-ones pontuais, mas sessões de acompanhamento onde possamos discutir problemas de gestão de pessoas, dificuldades de comunicação com membros da equipe e estratégias para melhorar o engajamento. Ter um canal mais estruturado de comunicação ajudaria os AGES IV a lidarem melhor com situações complexas de gestão.

\indent Encerro este memorial com a certeza de que a AGES foi uma experiência extraordinária e extremamente enriquecedora. Os desafios enfrentados, os aprendizados conquistados e as amizades construídas ao longo dessas quatro fases me prepararam verdadeiramente para o mercado de trabalho. Agradeço sinceramente à professora Alessandra, cujas orientações e suporte foram fundamentais para o desenvolvimento dos projetos e para meu crescimento acadêmico e profissional. Levo comigo as lições aprendidas tanto nos sucessos quanto nos desafios e me sinto mais preparado para os próximos passos da minha carreira em Engenharia de Software. Saio da AGES com a certeza de que sou capaz de liderar projetos complexos, adaptar-me a diferentes contextos, aprender continuamente e, acima de tudo, entregar valor genuíno para clientes e para a sociedade. A responsabilidade de ser AGES IV me ensinou que tecnologia é sobre pessoas e impacto real, e essa lição carregarei para toda minha carreira profissional.