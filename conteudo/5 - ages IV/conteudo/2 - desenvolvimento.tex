\section[Desenvolvimento do Projeto]{Desenvolvimento do Projeto}

\subsection{Repositório do Código Fonte do Projeto}
\indent Para este projeto, nós estruturamos o código-fonte em dois repositórios principais: um para o frontend e outro para o backend. Optamos por uma abordagem de web responsivo para o frontend, garantindo que a aplicação seja totalmente funcional em dispositivos móveis sem a necessidade de um repositório ou aplicativo nativo separado. Durante o desenvolvimento, enfrentamos um problema crítico de rollback no GitLab \cite{gitlab} da \ac{pucrs}, o que nos levou a tomar a decisão técnica de migrar o projeto para o GitHub \cite{github}, a fim de garantir maior estabilidade e controle de versionamento. Por essa razão, mantemos ambos os conjuntos de repositórios. A documentação técnica que orientou nosso desenvolvimento também foi duplicada em ambas as plataformas. Os links para acessar cada um desses repositórios são:

{\raggedright
\url{https://tools.ages.pucrs.br/gestao-de-planos-de-contingencia-em-desastres/hopeful-backend} - Link para o repositório do Código Fonte do Projeto referente ao backend (GitLab)

\url{https://tools.ages.pucrs.br/gestao-de-planos-de-contingencia-em-desastres/hopeful-frontend} - Link para o repositório do Código Fonte do Projeto referente ao frontend (GitLab)

\url{https://tools.ages.pucrs.br/gestao-de-planos-de-contingencia-em-desastres/hopeful-wiki/-/wikis/home} - Link para o repositório do Código Fonte do Projeto referente à wiki (GitLab)

\url{https://github.com/Hopeful-ages/backend} - Link para o repositório do Código Fonte do Projeto referente ao backend (GitHub)

\url{https://github.com/Hopeful-ages/frontend} - Link para o repositório do Código Fonte do Projeto referente ao frontend (GitHub)

\url{https://github.com/Hopeful-ages/wiki/wiki} - Link para o repositório do Código Fonte do Projeto referente à wiki (GitHub)
\par}

\subsection{Banco de Dados Utilizado}
\indent Para o banco de dados principal do nosso projeto, nós escolhemos o PostgreSQL \cite{postgresql}, um sistema de gerenciamento de banco de dados relacional (\ac{sgbd}) de código aberto. A escolha se deu por sua robustez, confiabilidade e alto desempenho, além de sua compatibilidade com o padrão \ac{sql} e suporte a tipos de dados complexos (como \ac{uuid}, que utilizamos em nossas chaves primárias).

\indent Para os testes de integração, nós utilizamos o H2 \cite{h2database}, um banco de dados relacional leve e em memória. Essa abordagem nos permitiu validar a interação entre os diferentes componentes do software em um ambiente de teste rápido e isolado, sem a necessidade de um servidor de banco de dados físico.

\indent Como framework de persistência, adotamos a especificação \ac{jpa} \cite{jpa}. O uso do \ac{jpa} nos permitiu abstrair a complexidade do \ac{sql} e manipular os dados diretamente como objetos Java \cite{java}, facilitando o desenvolvimento. Como implementação concreta do \ac{jpa}, utilizamos o Hibernate \cite{hibernate}. Na Figura 24, apresento o diagrama do banco de dados que elaboramos para o projeto.

\begin{figure}[H]
    \centering
    \caption{Diagrama do banco de dados do projeto}
    \includegraphics[width=1\linewidth]{conteudo//5 - ages IV//conteudo//figures/banco-de-dados-esquema-conceitual-ages4}
    
    Fonte: \url{https://tools.ages.pucrs.br/gestao-de-planos-de-contingencia-em-desastres/hopeful-wiki/-/wikis/Banco%20de%20Dados}
    \label{fig:banco-de-dados-esquema-conceitual-ages4}
\end{figure}

\subsection{Arquitetura Utilizada}
\indent Para o desenvolvimento do nosso projeto, nós adotamos o padrão arquitetural front-back, separando as responsabilidades do sistema em duas aplicações distintas: um backend desenvolvido com Spring Boot \cite{spring} e Java 21 \cite{java}, e um frontend com Next.js \cite{nextjs}. Na Figura 25, apresento o diagrama da arquitetura completa do sistema:

\begin{figure}[H]
    \centering
    \caption{Diagrama da arquitetura do sistema}
    \includegraphics[width=1\linewidth]{conteudo//5 - ages IV//conteudo//figures/arquitetura-de-sistema-ages4.png}
    
    Fonte: \url{https://tools.ages.pucrs.br/gestao-de-planos-de-contingencia-em-desastres/hopeful-wiki/-/wikis/arquitetura}
    \label{fig:arquitetura-sistema-ages4}
\end{figure}

\indent \textbf{Arquitetura do Backend:} No backend, nós estruturamos o projeto em camadas, seguindo uma arquitetura em camadas que separa as responsabilidades de forma clara e organizada. O fluxo de dados segue a seguinte hierarquia:

\begin{itemize}
    \item \textbf{Controller}: Camada de entrada que recebe as requisições HTTP do frontend. Os controllers são responsáveis por validar os dados de entrada, delegar o processamento para a camada de serviço e retornar as respostas adequadas.
    
    \item \textbf{Service}: Contém toda a lógica de negócio da aplicação. Esta camada processa as regras de negócio, orquestra as operações entre diferentes entidades e coordena as transações.
    
    \item \textbf{Entity Mapper}: Responsável pela conversão entre \ac{dto} (Data Transfer Objects) e entidades do banco de dados. Essa camada garante que os dados trafeguem no formato adequado entre as diferentes camadas da aplicação.
    
    \item \textbf{Entities}: Representam os modelos de dados que são mapeados diretamente para as tabelas do banco de dados, utilizando anotações do \ac{jpa} \cite{jpa}.
    
    \item \textbf{Repository}: Camada de acesso aos dados que abstrai as operações de persistência. Utiliza o Spring Data JPA para realizar operações de \ac{crud} e consultas customizadas ao banco de dados.
    
    \item \textbf{DTO}: Objetos de transferência de dados que trafegam entre o backend e o frontend, garantindo que apenas os dados necessários sejam expostos pela \ac{api}.
    
    \item \textbf{Database}: Banco de dados PostgreSQL \cite{postgresql} que armazena de forma persistente todas as informações do sistema.
\end{itemize}

\indent \textbf{Arquitetura do Frontend:} No frontend, nós desenvolvemos a aplicação utilizando React \cite{react} com TypeScript \cite{typescript}, sobre o framework Next.js \cite{nextjs}. Nossa arquitetura frontend segue uma estrutura componentizada e modular:

\begin{itemize}
    \item \textbf{View}: Representa as páginas da aplicação, responsáveis por renderizar a interface do usuário e gerenciar o estado global das telas.
    
    \item \textbf{Components}: Componentes React reutilizáveis que compõem as interfaces. Cada componente é responsável por uma parte específica da \ac{ui}, promovendo a reutilização de código.
    
    \item \textbf{Services}: Camada que gerencia toda a comunicação com o backend através de requisições HTTP. Os services abstraem as chamadas à \ac{api}, centralizando a lógica de comunicação.
\end{itemize}

\indent Nossa estilização foi feita inteiramente com Tailwind \ac{css} \cite{tailwind} para garantir um desenvolvimento ágil e responsivo, e utilizamos a biblioteca Lucide \cite{lucide} para os ícones, por ser moderna e personalizável. Essa arquitetura nos permitiu manter uma separação clara de responsabilidades, facilitando a manutenção, os testes e a escalabilidade do projeto.

\subsection{Protótipos das Telas Desenvolvidas}
\indent Para o desenvolvimento das telas deste projeto, nós seguimos cuidadosamente as diretrizes de design e a identidade visual já estabelecidas pelo cliente. A paleta de cores definida foi estritamente em preto e branco, e a fonte tipográfica solicitada foi a "Barlow". Além disso, nós integramos o header e o logotipo já existentes na landing page do cliente para garantir consistência entre as plataformas. O cliente também enfatizou a importância da padronização em toda a aplicação, como a uniformidade no tamanho dos inputs.

\indent Um requisito central do design foi que a nossa tela principal, a de cadastro de cenários, tivesse a aparência de uma "ficha", espelhando o formato do \ac{pdf} que o sistema gera ao final. Embora nossa equipe tivesse outras visões de design, nós nos alinhamos à preferência do cliente, compreendendo que na metodologia \ac{ages} o objetivo é atender às suas necessidades.

\indent Os protótipos completos desenvolvidos para o projeto, incluindo a tela de cadastro de cenário na visão do administrador, a tela de cadastro na visão do usuário de serviço e todas as demais interfaces da aplicação, podem ser visualizados no link do Figma \cite{figma} abaixo:

{\raggedright
Figma: \url{https://www.figma.com/design/CaQZSC2AycNI8dkSKO3XnB/Gestão-de-Desastres}
\par}

\subsection{Tecnologias Utilizadas}
\indent Para o desenvolvimento do nosso projeto, nós utilizamos um conjunto de tecnologias modernas e robustas. A seguir, detalho as principais ferramentas que compuseram nossa stack de desenvolvimento:

\begin{itemize}
    \item \textbf{Java 21} \cite{java}: Linguagem de programação principal que utilizamos para toda a lógica de negócio no lado do servidor.
    
    \item \textbf{Spring Boot} \cite{spring}: Framework que escolhemos para construir nossa aplicação de forma ágil e escalável.
    
    \item \textbf{PostgreSQL} \cite{postgresql}: Nosso sistema de gerenciamento de banco de dados relacional principal, escolhido por sua robustez e confiabilidade.
    
    \item \textbf{JPA (Hibernate)} \cite{jpa} \cite{hibernate}: Especificação e implementação que adotamos para o mapeamento objeto-relacional (\ac{orm}), facilitando a persistência de dados.
    
    \item \textbf{H2 Database} \cite{h2database}: Banco de dados em memória que usamos para a execução de testes de integração, garantindo um ambiente rápido e isolado.
    
    \item \textbf{Swagger} \cite{swagger}: Ferramenta que utilizamos para documentar e testar nossa \ac{api} \ac{restful}, facilitando a comunicação entre o backend e o frontend.
    
    \item \textbf{Docker} \cite{docker}: Plataforma que adotamos para containerizar nossa aplicação, garantindo a consistência entre os ambientes de desenvolvimento, teste e produção.
    
    \item \textbf{Next.js} \cite{nextjs}: Framework React que utilizamos para a construção da nossa aplicação web, aproveitando seus recursos de renderização e roteamento.
    
    \item \textbf{React} \cite{react}: Biblioteca principal para a criação de nossas interfaces de usuário reativas e componentizadas.
    
    \item \textbf{TypeScript} \cite{typescript}: Linguagem que adotamos em todo o frontend para adicionar tipagem estática, aumentando a segurança e a manutenibilidade do código.
    
    \item \textbf{Tailwind CSS} \cite{tailwind}: Nossa escolha para a estilização, permitindo a criação de interfaces modernas e responsivas de forma ágil através de classes utilitárias.
    
    \item \textbf{Lucide} \cite{lucide}: Biblioteca de ícones que utilizamos para garantir uma iconografia moderna e consistente na aplicação.
\end{itemize}
