\subsection{Sprint 3}
\indent Na Sprint 3, como AGES III, eu e meus colegas Henrique e Marcelo enfrentamos um dos maiores desafios técnicos do projeto: configurar o pipeline de \ac{cicd} no GitLab. Fiquei responsável por coordenar essa tarefa, e nosso objetivo era implementar um sistema de deploy automático que publicasse as atualizações na nossa instância \ac{ec2} a cada commit ou merge request na branch main. Além dessa responsabilidade de infraestrutura, eu também fiquei encarregado de uma tarefa crucial para a experiência do usuário: ajustar o questionário conforme as novas especificações dos stakeholders, o que incluía integrar uma funcionalidade de autocomplete para os campos de quilombos, cidades e estados.

\indent No meio da sprint, enfrentamos um problema crítico: um incidente resultou no rollback completo do projeto para uma versão de uma semana atrás, nos fazendo perder todo o progresso do período. Foi uma situação extremamente frustrante, que exigiu de toda a equipe uma dedicação extra para refazer o trabalho, mas conseguimos não apenas restaurar o que foi perdido, como também implementar melhorias para evitar problemas similares no futuro. Após essa adversidade, concluímos a reconfiguração do deploy automático e os ajustes no questionário.

\indent Além do rollback, também me deparei com outros desafios técnicos complexos. Um deles foi uma falha na lógica do backend do questionário, que estava invertida em relação ao que os stakeholders solicitaram. Outro erro crítico que descobri estava relacionado a campos condicionais que, no backend, estavam configurados como obrigatórios, gerando erros de validação que precisei investigar e corrigir.

\indent Esta sprint me ensinou uma lição valiosa sobre como nem sempre podemos confiar na estabilidade da arquitetura que construímos. Foi um lembrete da importância de sempre termos planos de contingência e backups robustos. Aprendi também que, mesmo atuando no frontend, preciso manter uma comunicação mais próxima com a equipe do backend. Tentar resolver os problemas de integração sozinho se mostrou muito mais difícil do que se eu tivesse colaborado com eles desde o início.

\indent Para os próximos passos, nosso foco será finalizar as User Stories pendentes, corrigir os bugs restantes e ajustar a responsividade de algumas telas. Fico muito feliz em notar que, apesar dos desafios, o cliente está extremamente satisfeito com nosso progresso. Trabalhar em um projeto com potencial de impacto social real tem sido muito gratificante, e a possibilidade de ele ser implementado fora do ambiente da AGES seria um reconhecimento incrível do trabalho de toda a nossa equipe.