\subsection{Sprint 1}
\indent Na Sprint 1, nosso foco como AGES IV foi traduzir o planejamento em tarefas executáveis para a equipe. Como líder da frente de backend, defini as features prioritárias, que incluíam o \ac{crud} de usuário e, crucialmente, todo o fluxo de autenticação via token JWT \cite{jwt}, um pilar de segurança que considero indispensável. Fiquei impressionado com a proatividade dos AGES II, Ravel e João, que entregaram a estrutura do banco de dados com uma rapidez notável, embora tenhamos identificado posteriormente a necessidade de refatorar os nomes das colunas de português para inglês, um débito técnico que planejamos resolver.

\indent Minha atuação foi intensamente focada na mentoria. Acompanhei de perto a dupla Ravel e Matheus, que assumiu a complexa tarefa do JWT \cite{jwt}. No primeiro fim de semana, eles avançaram significativamente, mas encontraram alguns bugs que os impediram de finalizar. Em uma sessão de depuração com eles, identifiquei que o problema estava no uso de métodos depreciados da biblioteca JWT \cite{jwt} do Spring \cite{spring}. Além de corrigir o código, fiz questão de explicar a causa do erro para que o aprendizado fosse consolidado. Na sequência, eu e o André (AGES III) trabalhamos juntos para criar um sistema de proteção de rotas, garantindo que endpoints sensíveis só pudessem ser acessados por usuários autenticados.

\indent No frontend, também precisei atuar de perto. O Gabriel (AGES II) me procurou com dúvidas em sua tarefa, e percebi que o critério de aceitação que eu mesmo havia escrito não estava claro o suficiente. Pedi desculpas pela falha e o ajudei a entender o escopo, orientando-o também tecnicamente a usar o método PATCH para atualizações parciais, em vez de PUT. Com o avançar da sprint, notei que as tarefas de frontend estavam atrasando. Para não comprometer a entrega, tomei a iniciativa de intervir: implementei toda a integração da tela de login com o backend e realizei uma refatoração completa na estrutura de pastas do Next.js \cite{nextjs}, que não seguia os padrões do framework. Essa intervenção resultou em uma branch com muitas mudanças, que revisei detalhadamente com o André antes de integrar à dev.

\indent A sprint também exigiu agilidade para lidar com imprevistos. Perto da entrega, percebemos que faltava um endpoint para a busca de cidades, uma task que não havia sido mapeada. Assumi a responsabilidade e a desenvolvi rapidamente. Em paralelo, todos os AGES III e IV estávamos prospectando provedores de hosting alternativos à AWS \cite{aws}, uma tarefa liderada pelo Denilson, enquanto os demais ajudavam a equipe. Conseguimos entregar tudo o que planejamos, mas isso exigiu um esforço extra de programação dos AGES IV na reta final, algo que identifiquei como um sintoma de que a equipe, por ser nova, ainda estava se adaptando às tecnologias.

\indent Na apresentação, o Abner aprovou as funcionalidades com poucos apontamentos de design. Nossa retrospectiva, com o tema criativo de Pokémon, foi um momento de reflexão profunda. Fiquei feliz por receber muitos feedbacks positivos ("estrelas") dos meus colegas, mas foquei em discutir os problemas que identifiquei. Apontei uma queda de motivação geral da equipe na metade final da sprint, com exceção de alguns membros, como o Ravel e o Matheus, que buscaram novas tarefas após concluírem as suas.

\indent Assumi total responsabilidade pelos critérios de aceitação que, por vezes, estavam ambíguos, e me comprometi a melhorar isso. O ponto mais crítico, para mim, foi a baixa adesão às dailies, especialmente na véspera da entrega. A comunicação é vital, e decidi que na próxima sprint serei mais rigoroso com a participação.

\indent Para a Sprint 2, o plano já está traçado: vamos focar no desenvolvimento do \ac{crud} de cenários, que é a funcionalidade central do backend. Também será a sprint em que testaremos a infraestrutura na Hostinger \cite{hostinger}. Será um período intenso, mas estou muito otimista com o potencial da nossa equipe e com os aprendizados que tivemos.