\chapter[Considerações Finais]{CONSIDERAÇÕES FINAIS}

\indent Este memorial reuniu as experiências e aprendizados vivenciados ao longo dos quatro projetos desenvolvidos na Agência Experimental de Engenharia de Software (AGES I a IV). Cada fase contribuiu de maneira singular para a construção de competências técnicas e interpessoais, consolidando minha formação e oferecendo uma visão abrangente sobre os processos e desafios inerentes à prática da Engenharia de Software.

\indent Ao refletir sobre minha trajetória completa, percebo uma evolução profissional e pessoal que superou minhas expectativas iniciais. Quando iniciei minha jornada na AGES, tinha apenas conhecimento teórico e pouca experiência prática. Hoje, concluo essa etapa como um profissional completo, capaz de liderar equipes, tomar decisões arquiteturais complexas e entregar soluções que geram impacto real. A AGES foi, sem dúvida, a disciplina mais transformadora do curso, ensinando-me que Engenharia de Software vai muito além de escrever código — trata-se de resolver problemas reais, trabalhar colaborativamente, gerenciar expectativas e entregar valor genuíno.

\indent Ao longo dos projetos, desenvolvi tanto competências técnicas quanto interpessoais. Consolidei o domínio de desenvolvimento backend e frontend, práticas de arquitetura de software, metodologias ágeis e ferramentas de DevOps. Mais importante ainda, exercitei soft skills fundamentais como gestão de tempo, comunicação em equipe, liderança e delegação de responsabilidades. A conciliação das atividades de AGES IV com 11 disciplinas simultâneas foi um dos maiores desafios de gestão de tempo que enfrentei, mas me preparou para lidar com múltiplas prioridades no ambiente profissional.

\indent Comparando o início da minha trajetória na AGES com o momento atual, percebo uma transformação significativa. De alguém que entendia Engenharia de Software como simplesmente escrever código, hoje compreendo que a disciplina é muito mais ampla. Engenharia de Software é sobre pessoas — entender necessidades de clientes, comunicar-se efetivamente com equipes, gerenciar expectativas e construir soluções que geram impacto real. Os projetos que mais me marcaram foram aqueles com impacto social claro: a Cartilha Interativa de Saúde Bucal para Comunidades Quilombolas e o Hopeful, que poderia salvar vidas em desastres naturais. Esses projetos transformaram minha percepção sobre o propósito da tecnologia e confirmaram meu desejo de trabalhar com soluções que façam diferença real.

\indent A AGES representa uma oportunidade ímpar para estudantes aplicarem conceitos em contextos reais. Como sugestões de melhoria para a disciplina, destaco dois pontos principais. Primeiro, a \textbf{padronização dos entregáveis}: descobri apenas na última sprint que havia artefatos específicos de gerenciamento (Plano de Comunicação e Plano de Riscos) que deveriam ser elaborados. Sugiro que a disciplina estabeleça claramente os entregáveis obrigatórios para cada fase e comunique logo no início do semestre, evitando surpresas e permitindo melhor planejamento.

\indent Segundo, a \textbf{participação dos AGES IV na avaliação dos alunos}: como líder de grupo, acompanhei de perto o desempenho de cada membro da equipe. Acredito que os AGES IV poderiam estar envolvidos no processo de avaliação, não para decidir a nota final, mas para participar de uma votação ou fornecer parecer detalhado sobre o desempenho individual. Essa dinâmica daria maior responsabilidade aos AGES IV e tornaria a avaliação mais justa, pois temos visão privilegiada do comprometimento e evolução de cada aluno.

\indent Encerro este memorial com a certeza de que a AGES foi uma experiência extraordinária. Os desafios enfrentados, os aprendizados conquistados e as amizades construídas me prepararam verdadeiramente para o mercado de trabalho. Agradeço sinceramente à professora Alessandra, cujas orientações foram fundamentais para o desenvolvimento dos projetos e para meu crescimento acadêmico e profissional. Saio da AGES com a certeza de que sou capaz de liderar projetos complexos, adaptar-me a diferentes contextos e entregar valor genuíno para clientes e para a sociedade. A responsabilidade de ser AGES IV me ensinou que tecnologia é sobre pessoas e impacto real, e essa lição carregarei para toda minha carreira profissional.