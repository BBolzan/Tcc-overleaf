\subsection{Sprint 2}
\indent Na Sprint 2, nosso projeto avançou significativamente, com o planejamento de quatro User Stories fundamentais (US 3, 4, 5 e 8). Além do desenvolvimento dessas funcionalidades, tínhamos como meta crucial realizar o primeiro deploy da aplicação. Para isso, eu e os outros AGES III precisávamos finalizar o diagrama de deploy, calcular os custos da AWS \cite{aws} e formalizar o pedido da nossa instância \ac{ec2}. Meu squad, o "ExaltaSamba", ficou responsável pela US 3, focada no desenvolvimento do questionário interativo. Como AGES III, eu também assumi a responsabilidade de liderar o processo de deploy, uma tarefa que exigiu tanto coordenação técnica quanto conhecimento de infraestrutura em nuvem.

\indent Com o apoio essencial do meu time, consegui concluir o desenvolvimento do formulário do questionário. Nós trabalhamos de forma colaborativa para implementar toda a estrutura, validações e funcionalidades conforme os requisitos, resultando em um questionário intuitivo e funcional. Em paralelo, finalizei a complexa tarefa de deploy com o auxílio fundamental do AGES IV Marcos. Juntos, realizamos a configuração detalhada do ambiente na AWS \cite{aws}, preparamos os arquivos, implementamos as políticas de segurança e concluímos com sucesso a primeira versão em produção.

\indent Durante esta sprint, enfrentei alguns desafios significativos. O principal foi a gestão do tempo, pois tive dificuldades em conciliar as intensas demandas do projeto com outras atividades acadêmicas, o que me levou a realizar algumas entregas próximo ao prazo final. Outro obstáculo foi um erro no provisionamento automático das permissões na AWS \cite{aws}, que nos atrasou até que os acessos fossem corrigidos. Tecnicamente, também enfrentei dificuldades na integração do formulário, com pequenos erros de digitação e tipagem nos nomes dos campos que geraram bugs e demandaram um tempo considerável para a depuração.

\indent Contudo, as lições aprendidas foram extremamente valiosas. Percebi a importância de um planejamento de tempo mais eficiente para evitar a pressão de última hora. Além disso, adquiri um conhecimento técnico profundo sobre o processo de deploy na AWS \cite{aws}, incluindo a configuração de servidores \ac{ec2}, gestão de permissões IAM, os primeiros passos em \ac{cicd} e o cálculo de custos de infraestrutura — habilidades que serão muito úteis em minha carreira.

\indent Para a próxima sprint, pretendo focar em aprimorar a documentação do nosso processo de deploy e evoluir nosso pipeline de automação. Também planejo contribuir ativamente nas User Stories restantes e trabalhar na otimização do ambiente de produção para melhorar o desempenho e reduzir custos.