\section[Introdução]{Introdução}
\indent Meu próximo projeto foi a "Cartilha Interativa de Saúde Bucal para Comunidades Quilombolas", uma iniciativa solicitada pelos stakeholders Márcio Lima Grossi (Odontologia/PUCRS) e Patrícia Krieger Grossi (Serviço Social/PUCRS), que contam também com a colaboração de duas alunas da Odontologia. O nosso principal objetivo foi desenvolver uma cartilha online para atender às necessidades específicas de saúde bucal desta comunidade historicamente vulnerabilizada.

\indent A principal justificativa para o projeto é que, embora existam diversos materiais sobre saúde bucal, nenhum deles é direcionado às comunidades quilombolas do Rio Grande do Sul. Essas comunidades, que representam a maior concentração do grupo na região sul, enfrentam dificuldades no acesso a políticas públicas de saúde e outros desafios sociais.

\indent Diante desse cenário, nós nos dedicamos a criar uma plataforma web que fosse além do acesso à informação. Nosso objetivo foi desenvolver uma ferramenta que permitisse a coleta de dados por meio de questionários, oferecendo pré-diagnósticos e recomendações personalizadas. Com essa solução, buscamos possibilitar um melhor entendimento da autopercepção de saúde oral nessas comunidades, considerando tanto fatores clínicos quanto sociodemográficos.

\indent Para este projeto, contamos com a orientação do professor Dilnei Venturini, que nos auxiliou na comunicação com os stakeholders e no desenvolvimento do relatório final. Nós desenvolvemos o projeto entre janeiro e junho de 2025, com a equipe de desenvolvimento apresentada na Figura 20.

\begin{figure}[H]
    \centering
    \caption{Fotografia de todos os participantes do projeto}
    \includegraphics[width=1\linewidth]{conteudo//4 - ages III//conteudo//figures/foto-time-ages3.jpg}
    
    Fonte: \url{https://tools.ages.pucrs.br/saude-bucal-quilombola/wiki/-/wikis/home}
    \label{fig:foto-time-ages3}
\end{figure}