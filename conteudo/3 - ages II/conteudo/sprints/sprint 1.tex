\subsection{Sprint 1}
\indent Na Sprint 1, nossa equipe deu continuidade ao desenvolvimento do Projeto Planline. Minha responsabilidade nesta sprint foi a task 12, focada no desenvolvimento móvel, que consistia na criação da tela de carregamento (loading) e da tela inicial (home) do aplicativo. Eu desenvolvi essas interfaces em colaboração com meu colega de equipe, Eduardo Garcia. Trabalhando em parceria, conseguimos implementar e finalizar ambas as telas. Os resultados do nosso trabalho nesta tarefa são apresentados na Figura 14.

\begin{figure}[H]
    \centering
    \small
    \caption{Telas de carregamento e inicial desenvolvidas no mobile}
    \includegraphics[width=1\linewidth]{conteudo//3 - ages II//conteudo//figures/sprint-1-imagem-1-ages2.png}
    Fonte: Figma do Projeto
    \label{fig:sprint-1-imagem-1-ages2}
\end{figure}

\indent O desenvolvimento dessas telas representou um desafio técnico para mim, pois eu ainda não estava familiarizado com o uso do Tailwind CSS \cite{css}. Para superar essa barreira, precisei primeiro aprofundar meus conhecimentos nos fundamentos do CSS \cite{css} nativo para, em seguida, conseguir aplicar e adaptar esses conceitos ao framework Tailwind \cite{tailwind}. Foi uma importante curva de aprendizado.

\indent Além do meu trabalho no desenvolvimento frontend, eu também colaborei com os outros membros AGES II da equipe na modelagem do banco de dados relacional em PostgreSQL \cite{postgresql}. Essa foi uma etapa fundamental do projeto, e nossa dedicação a ela foi crucial para garantir que a estrutura de dados funcionasse de forma eficiente e atendesse aos requisitos dos usuários.