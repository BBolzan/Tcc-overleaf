\section[Conclusão]{Conclusão}

\indent Ao refletir sobre minha trajetória como AGES IV no projeto Hopeful, posso afirmar que esta foi a experiência mais desafiadora e gratificante do meu percurso acadêmico. Cursando AGES IV simultaneamente com outras 11 disciplinas — pois deixei todas as eletivas para o último semestre — enfrentei um desafio de gestão de tempo sem precedentes. A pressão se intensificou na última sprint, quando precisei tirar férias do trabalho para me concentrar nas demandas do projeto e das outras disciplinas. Apesar dessa sobrecarga, consegui desempenhar bem o papel de líder de grupo e AGES IV, demonstrando comprometimento que superou minhas expectativas.

\indent Tecnicamente, atuei nas tarefas mais complexas, como a implementação do JWT \cite{jwt}, integração de autenticação, refatoração da estrutura do Next.js \cite{nextjs}, geração de \ac{pdf} com Thymeleaf \cite{thymeleaf}, e configuração de HTTPS usando Let's Encrypt \cite{letsencrypt} e Certbot \cite{certbot}. Muitas vezes, quando percebia que as entregas estavam em risco, assumia a responsabilidade de implementar o que faltava, garantindo a satisfação do cliente e da equipe.

\indent No que se refere às soft skills, recebi feedbacks positivos sobre meu empenho, domínio do projeto e planejamento. Fui responsável pela gestão dos próprios AGES IV, definindo responsabilidades conforme solicitado pela professora Alessandra. Porém, minha gestão de projeto apresentou falhas. No início, tentava fazer tudo sozinho, não envolvendo suficientemente os outros AGES IV. Felizmente, após o feedback da professora, consegui melhorar e dividir melhor as responsabilidades. Outro ponto crítico foi a gestão de artefatos — descobri os artefatos obrigatórios de AGES IV apenas na última sprint, gerando trabalho extra. A qualidade das tasks no início também poderia ter sido melhor, com critérios de aceitação ambíguos que geraram confusão.

\indent Um problema significativo foi não ter envolvido mais a professora Alessandra quando minha cobrança sozinha não funcionava. Membros da equipe demoravam para entregar tasks ou subiam pull requests sem descrição e não respondiam quando chamados. Percebi tarde que deveria ter escalado essas questões, pois apenas a professora teria autoridade para resolvê-las efetivamente. A documentação da Wiki também ficou em segundo plano, prejudicando a rastreabilidade de informações.

\indent Este foi, sem dúvida, o melhor projeto que trabalhei. Não apenas pela qualidade técnica — sistema completo com autenticação JWT \cite{jwt}, geração de \ac{pdf}, PostgreSQL \cite{postgresql}, responsividade mobile, 91\% de cobertura de testes no backend, recursos de acessibilidade e deploy em produção — mas especialmente pelo impacto social. Conversar com o Abner e entender como a aplicação pode ajudar a salvar vidas em desastres naturais, como os que ocorreram no Sul do Brasil, transformou minha percepção sobre o propósito da tecnologia. Saber que cidades poderiam ter planos de contingência melhor estruturados graças ao nosso trabalho me fez perceber que estávamos construindo algo verdadeiramente importante.

\indent Esse senso de propósito intensificou minha dedicação de forma única. Ao assumir a responsabilidade máxima como AGES IV, comecei a me importar profundamente com cada detalhe do projeto. Cada funcionalidade incompleta, cada bug, cada detalhe mal acabado passou a me incomodar pessoalmente, motivando-me a fazer "horas extras" de código para garantir a melhor entrega possível ao cliente.

\indent Fazendo uma autoavaliação, considero que me saí muito bem como AGES IV. Equilibrei pontos fortes — liderança técnica, proatividade, capacidade de resolver problemas complexos, comprometimento com a qualidade e impacto positivo no time — com as áreas identificadas para melhoria, como gestão de artefatos e envolvimento mais efetivo da professora em situações específicas. Levo essas lições como aprendizado para experiências futuras.