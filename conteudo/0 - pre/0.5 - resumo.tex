\begin{resumo}[\protect\bfseries Resumo]
  Neste documento, está o relato da minha experiência e vivência na cadeira AGES
(Agência Experimental de Engenharia de Software), a qual tem o objetivo de preparar
o aluno para o mercado de trabalho, simulando um método ágil, além de muitas tarefas
em que o aluno irá viver no dia a dia sendo um profissional da Engenharia de Software.
A cadeira é dividida em quatro etapas (quatro semestres) sendo a primeira etapa,
seguindo o currículo como recomendado, no terceiro período do curso Engenharia de
Software da faculdade PUCRS (Pontifícia Universidade Católica do Rio Grande do
Sul). A primeira etapa é a AGES I, na qual o aluno atua principalmente na
programação do projeto, nos mockups e na documentação. Já o aluno que é AGES
II, atua diretamente no desenvolvimento do banco de dados e nas mesmas funções
que o AGES anterior. Já na AGES III, ele vaificar responsável por toda a arquitetura
do projeto, na escolha das tecnologias, juntamente com o time e responsável em
ajudar os outros em relação a configuração em suas respectivas máquinas. Por último
e não menos importante, temos o AGES IV, que exerce a responsabilidade de ser o
gerente do projeto, falar com os stakeholders, além de participar diretamente na
gestão da equipe
  \bigbreak\
  \\\textbf{PALAVRAS CHAVES:}
  AGES, Engenharia de Software. Responsabilidade,
Arquitetura do projeto, Programação, Desenvolvimento.
\end{resumo}